% file: .../doc/c++-gen.tex

% $Header: /cvs/Darwin/Security/SecuritySNACCRuntime/doc/c++-gen.tex,v 1.1.1.1 2001/05/18 23:14:10 mb Exp $
% $Log: c++-gen.tex,v $
% Revision 1.1.1.1  2001/05/18 23:14:10  mb
% Move from private repository to open source repository
%
% Revision 1.1.1.1  1999/03/16 18:05:51  aram
% Originals from SMIME Free Library.
%
% Revision 1.1  1997/01/01 22:47:35  rj
% first check-in
%

\chapter{\label{c++-code-gen-chapter}C++ Code Generation}

\section{\label{intro-C++-section}Introduction}

The C++ backend of snacc was designed after the C backend had been
written.  The basic model that the generated C++ uses is similar to
that of the generated C, but benefits from the object oriented features of
C++.  This was my first real foray into C++ which may be evident from
some of the design.

As with C, two files are generated for each ASN.1 module, a {\ufn .C}
and a {\ufn .h} file.

Some cleaner designs were rejected either due to their poor
performance or the inability of the available C++ compiler to handle
those features.

Tags and lengths would fit nicely into their own classes but
performance was considerably worse than the technique used in the C
environment.  The C design was retained in the C++ model for its
superior performance.

For error management C++'s {\C try} and {\C throw} are obvious
replacements for the {\C setjmp} and {\C longjmp} used by the C
decoders. Unfortunately this is a newer C++ feature and is not yet
supported by gcc.

C++ templates are very attractive for type safe lists (for SET OF and
SEQUENCE OF) without duplicating code.  Template support was shaky in
gcc at the time the generated code was being tested so they were
rejected. Instead, each list generates its own new class with all of
the standard list routines.

As with the C code generation chapter, we will use the EX1 module to
help illustrate some of the code generation.  The following is the
same EX1 module used in the C section.

\begin{ASNcode}
EX1 DEFINITIONS ::=\\
BEGIN\\
\\
anOidVal      OBJECT IDENTIFIER ::= \{ joint-iso-ccitt 40 foobar(29) \}\\
theSameOidVal OBJECT IDENTIFIER ::= \{ 2 40 29 \}\\
anIntVal      INTEGER ::= 1\\
aBoolVal      BOOLEAN ::= TRUE\\
\\
T1 ::= SEQUENCE\\
\{\+\\
  INTEGER OPTIONAL,\\
  OCTET STRING OPTIONAL,\\
  ENUMERATED \{ a(0), b(1), c(2) \},\\
  SEQUENCE OF INTEGER,\\
  SEQUENCE \{ id OBJECT IDENTIFIER, value OCTET STRING \},\\
  CHOICE \{ INTEGER, OBJECT IDENTIFIER \}\-\\
\}\\
\\
END
\end{ASNcode}

The C++ backend to snacc is in the {\ufn \dots/compiler/back-ends/c++-gen/} directory if you want to alter it.

\section{\label{naming-C++-section}ASN.1 to C++ Naming Conventions}

The C++ name for a type or value is the same as its ASN.1 name with
any hyphens converted to underscores.

When an ASN.1 type or value name (after converting any hyphens to
underscores) conflicts with a C++ keyword or the name of a type in
another ASN.1 module (name clashes within the same ASN.1 scope are
considered errors and are detected earlier), the resulting C++ class
name will be the conflicting name with digits appended to it.

Empty field names in SETs, SEQUENCEs, and CHOICEs will be filled.  The
field name is derived from the type name for that field. The library
types such as INTEGER etc. have default field names defined by the
compiler (see {\ufn \dots/compiler/back-ends/c-gen/rules.c} and
{\ufn \dots/compiler/back-ends/c++-gen/rules.c}).
The first letter of the field name is in lower case.
Empty field names should be fixed properly by adding them to the ASN.1 source.

New type definitions will be generated for SETs, SEQUENCEs, CHOICEs,
ENUMERATED, INTEGERs with named numbers and BIT STRINGs with named bits
whose definitions are embedded in other SET, SEQUENCE, SET OF,
SEQUENCE OF, or CHOICE definitions.  The name of the new type
is derived from the name of the type in which it was embedded and will
be made unique by appending digits if necessary.

\section{\label{type-gen-C++-section}ASN.1 to C++ Class Translation}

This section describes how C++ classes are used to represent each
ASN.1 type.  First, the general characteristics of each ASN.1 type's
C++ class will be discussed followed by how the aggregate types (SETs,
SEQUENCEs, CHOICEs, SET OFs, and SEQUENCE OFs) are represented.  The
representations of non-aggregate types (INTEGER, BOOLEAN, OCTET
STRING, BIT STRING, OBJECT IDENTIFIER) and ANY and ANY DEFINED BY
types are presented in the next chapter since they form part of the
C++ ASN.1 runtime library.

Every ASN.1 type is represented by a C++ class with the following
characteristics:
\newcounter{saveenumi}
\begin{enumerate}
  \item it inherits from the {\C AsnType} base class
  \item it has a parameterless constructor
  \item it has a copy constructor
  \item it has a destructor
  \item it has a clone method, {\C Clone}
  \item it has an assignment operator
  \item it has a content encode and decode method, {\C BEncContent} and {\C BDecContent}
  \item it has a PDU encode and decode method, {\C BEnc} and {\C BDec}
  \item it has a top level interfaces to the PDU encode and decode methods (handles the {\C setjmp} etc.) for the user, {\C BEncPdu} and {\C BDecPdu}
  \item it has a print method, {\C Print}, a virtual function that gets called from a global <\/<-operator
  \setcounter{saveenumi}{\value{enumi}}
\end{enumerate}
If the metacode has been enabled:
\begin{enumerate}
  \setcounter{enumi}{\value{saveenumi}}
  \item it has a virtual function {\C \_getdesc} that returns the classes meta description% (only if metacode is enabled)
  \item if it is a structured type, it has a virtual function {\C \_getref} that returns a pointer to one of its components/members, specified through its name% (only if metacode is enabled)
  \setcounter{saveenumi}{\value{enumi}}
\end{enumerate}
If the Tcl code has been enabled:
\begin{enumerate}
  \setcounter{enumi}{\value{saveenumi}}
  \item it has a virtual function {\C TclGetDesc} to access the metacode's {\C \_getdesc} routine from Tcl% (only if Tcl code is enabled)
  \item it has a virtual function {\C TclGetVal} to retrieve an instance's value% (only if Tcl code is enabled)
  \item it has a virtual function {\C TclSetVal} to change an instance's value% (only if Tcl code is enabled)
  \item for SET, SEQUENCE, SET OF and SEQUENCE of: it has a virtual function {\C TclUnsetVal} to clear OPTIONAL members or to delete list elements, respectively% (only if Tcl code is enabled)
\end{enumerate}

The following C++ fragment shows the class features listed above in greater
detail.
\begin{Ccode}
class Foo: public AsnType\\
\{\+\\
  \dots // data members\\
\\
\<public:\\
          \>\>Foo();\\
          \>\>Foo (const Foo \&);\\
          \>\>\~{}Foo();\\
  AsnType \>\>*Clone() const;\\
  Foo     \>\>\&operator = (const Foo \&);\\
\\
  // content encode and decode routines\\
  AsnLen  \>\>BEncContent (BUF\_TYPE b);\\
  void    \>\>BDecContent (BUF\_TYPE b, AsnTag tag, AsnLen elmtLen,\\
                     \`AsnLen \&bytesDecoded, ENV\_TYPE env);\\
\\
  // PDU (tags/lengths/content) encode and decode routines\\
  AsnLen  \>\>BEnc (BUF\_TYPE b);\\
  void    \>\>BDec (BUF\_TYPE b, AsnLen \&bytesDecoded, ENV\_TYPE env);\\
\\
  // methods most likely to be used by your code.\\
  // Returns non-zero for success\\
  int     \>\>BEncPdu (BUF\_TYPE b, AsnLen \&bytesEncoded);\\
  int     \>\>BDecPdu (BUF\_TYPE b, AsnLen \&bytesDecoded);\\
\\
  void    \>\>Print (ostream \&os) const;\\
\\
\<\#if META\\
  const AsnTypeDesc  \>\>*\_getdesc() const;\\
  AsnType            \>\>*\_getref (const char *membername, bool create = false);\\
\<\#if TCL\\
  int                \>\>TclGetDesc (Tcl\_DString *) const;\\
  int                \>\>TclGetVal (Tcl\_Interp *) const;\\
  int                \>\>TclSetVal (Tcl\_Interp *, const char *valstr);\\
  int                \>\>TclUnsetVal (Tcl\_Interp *, const char *membername);\\
\<\#endif\\
\<\#endif\\
\<\};
\end{Ccode}

{\C BEnc} and {\C BDec} are PDU encode and decode methods.
{\C BEnc} encodes the tag and length pairs for the object's type as
well as the content (the object's value) to the given buffer,
{\C b}, and returns the number of bytes written to the buffer for
the encoding.

{\C BDec} decodes the expected tag and length pairs as well as the
content of the object it is invoked upon from the given buffer,
{\C b}, and increments {\C bytesDecoded}  by the byte length of
the tag(s), length(s) and value decoded.  The {\C env} parameter
will be used with {\C longjmp} if any decoding error occurs.
Decoding errors can be reported via {\C longjmp} from any of the
routines that {\C BDec} calls, such as {\C BDecContent};
{\C BDec} will call {\C longjmp} directly if the value does not
have the correct tag(s).

{\C BEncContent} and {\C BDecContent} only deal with the content
of the type their object represents.  {\C BEncContent} encodes the
object's value to the given buffer, {\C b}.

{\C BDecContent} decodes the object's value from the given buffer,
{\C b}.  The last tag and length pair on the content must be passed in
via the {\C tag} and {\C elmtLen} parameters.  The {\C tag},
although always present, will only be used when decoding OCTET STRING and
BIT STRING related types, to determine whether the encoding is
constructed.  The {\C elmtLen} is the length of the content and may
be the indefinite length form.  {\C bytesDecoded} is incremented by
the actual number of bytes in the content; this is normally the same
as {\C elmtLen} unless the indefinite length form was decoded.  The
{\C env} parameter will be used with {\C longjmp} if any decoding
error occurs.  The possible decoding errors depend on the type that is
being decoded.

{\C BEncPdu} and {\C BDecPdu} are top-level interfaces to the PDU
encode and decode routines.  They present the simplest interface;
they return TRUE if the operation succeeded and FALSE if an error
occurred.   Note that the {\C BDecPdu} routine sets up the
{\C env} parameter using {\C setjmp} for any {\C longjmp} calls
that may occur.  If you call {\C BDec} or {\C BDecContent}
directly from your code, you must use {\C setjmp} to setup the
{\C env} parameter.  {\C BEncPdu} checks for any buffer writing
errors and {\C BDecPdu} checks for any buffer reading errors.

The {\C Print} method prints the object's value in ASN.1 value
notation.  When printing SETs and SEQUENCEs, a global variable
is used for the current indent.

The {\C AsnType} base class, parameterless constructor and
{\C Clone} method are required by the ANY and ANY DEFINED BY type
handling mechanism explained in Sections \ref{asntype-C++-section} and
\ref{any-C++-section}.  In brief, the {\C AsnType} provides a base type
that has virtual {\C BEnc}, {\C BDec} and {\C Clone} routines.
The {\C Clone} routine is used to generate a new instance (not a
copy) of the object that it is invoked on.  This allows the ANY
DEFINED BY type decoder to create a new object of the correct type
from one stored in a hash table, when decoding (the {\C Clone}
routine calls the parameterless constructor).  The virtual {\C BEnc}
and {\C BDec} are called from {\C AsnAny} {\C BEnc} and
{\C BDec} methods.

The meta routines and the Tcl interface will be described in chapters \ref{meta-chapter} and \ref{tcl-if-chapter}, respectively.

\subsection{\label{C++-set-seq-section}SET and SEQUENCE}

SET and SEQUENCE types generate classes that have their components as
public data members.  This makes accessing the components similar to
referencing the fields of a C struct.  For example the {\C T1} type in
module EX1 will produce the following C++ class:

\begin{Ccode}
class T1: public AsnType\\
\{\\
public:\+\\
  AsnInt    \>\>*integer;\\
  AsnOcts   \>\>*octs;\\
  T1Enum    \>\>t1Enum;\\
  T1SeqOf   \>\>t1SeqOf;\\
  T1Seq     \>\>*t1Seq;\\
  T1Choice  \>\>*t1Choice;\\
\\
            \>\>T1();\\
%  {\\
%    /* init optional/default elements to NULL */\\
%    integer = NULL;\\
%    octs = NULL;\\
%  }\\
	    \>\>T1 (const T1 \&);\\
	    \>\>\~{}T1();\\
%  AsnType *Clone() { return new T1; }\\
  AsnType   \>\>*Clone() const;\\
\\
  T1        \>\>\&operator = (const T1 \&);\\
\\
  AsnLen    \>\>BEnc (BUF\_TYPE b);\\
  void      \>\>BDec (BUF\_TYPE b, AsnLen \&bytesDecoded, ENV\_TYPE env);\\
\\
  AsnLen    \>\>BEncContent (BUF\_TYPE b);\\
  void      \>\>BDecContent (BUF\_TYPE b, AsnTag tag, AsnLen elmtLen,\\
                         \`AsnLen \&bytesDecoded, ENV\_TYPE env);\\
\\
  int       \>\>BEncPdu (BUF\_TYPE b, AsnLen \&bytesEncoded);\\
  int       \>\>BDecPdu (BUF\_TYPE b, AsnLen \&bytesDecoded);\\
\\
  void      \>\>Print (ostream \&os) const;\\
\\
\<\#if META\\
  static const AsnSequenceTypeDesc     \\
				\>\>\_desc;\\
  static const AsnSequenceMemberDesc   \\
				\>\>mdescs[];\\
  const AsnTypeDesc             \>\>*\_getdesc() const;\\
  AsnType                       \>\>*\_getref (const char *membername, bool create = false);\\
\\
\<\#if TCL\\
  int                   \>\>TclGetDesc (Tcl\_DString *) const;\\
  int                   \>\>TclGetVal (Tcl\_Interp *) const;\\
  int                   \>\>TclSetVal (Tcl\_Interp *, const char *valstr);\\
  int                   \>\>TclUnsetVal (Tcl\_Interp *, const char *membname);\\
\<\#endif // TCL\\
\<\#endif // META\\
\<\};
\end{Ccode}

All OPTIONAL components in a SET or SEQUENCE are referenced by pointer.
The constructor will automatically set OPTIONAL fields to {\C NULL}\@.  The
other methods are as described at the beginning of this section.

SETs and SEQUENCEs must contain all non-OPTIONAL components and
SEQUENCEs must be ordered, otherwise an error is reported.  Tagging
errors are also reported.  All detected errors abort the decoding
process via {\C longjmp}.

\subsection{\label{C++-choice-section}CHOICE}

Each CHOICE type generates a class that has an anonymous union to hold the
components of the CHOICE and a {\C choiceId} field to indicate which
component is present.

Anonymous (un-named) unions allow you to reference the choice components
with just the field name of the component; this makes referencing the
contents of a CHOICE the same a referencing the contents of a SET or
SEQUENCE\@.

The {\C choiceId} field contains a value in the {\C ChoiceIdEnum}
that indicates the CHOICE field that is present.  The names in the
enumeration are derived from the field names of the CHOICE components.

When building a local value to be encoded, you must be sure to set the
{\C choiceId} such that it corresponds to the value in the union.  The
decoder will set the {\C choiceId} when decoding incoming values.

Tagging errors are reported and abort the decoding process via
{\C longjmp}.

The following C++ class is produced for the CHOICE in the EX1 module.

\begin{Ccode}
class T1Choice: public AsnType\\
\{\\
public:\+\\
  enum ChoiceIdEnum\\
  \{\+\\
    integerCid = 0,\\
    oidCid = 1\-\\
  \};\\
\\
  enum ChoiceIdEnum \>\>choiceId;\\
  union\\
  \{\+\\
    AsnInt   \>*integer;\\
    AsnOid   \>*oid;\-\\
  \};\\
\\
             \>\>T1Choice();\\
	     \>\>T1Choice (const T1Choice \&);\\
	     \>\>\~{}T1Choice();\\
  AsnType    \>\>*Clone() const;\\
\\
  T1Choice   \>\>\&operator = (const T1Choice \&);\\
\\
  AsnLen     \>\>BEncContent (BUF\_TYPE b);\\
  void       \>\>BDecContent (BUF\_TYPE b, AsnTag tag, AsnLen elmtLen,\\
                          \`AsnLen \&bytesDecoded, ENV\_TYPE env);\\
\\
  AsnLen     \>\>BEnc (BUF\_TYPE b);\\
  void       \>\>BDec (BUF\_TYPE b, AsnLen \&bytesDecoded, ENV\_TYPE env);\\
\\
  int        \>\>BEncPdu (BUF\_TYPE b, AsnLen \&bytesEncoded);\\
  int        \>\>BDecPdu (BUF\_TYPE b, AsnLen \&bytesDecoded);\\
\\
  void       \>\>Print (ostream \&os) const;\\
\\
\<\#if META\\
  static const AsnChoiceTypeDesc        \>\>\_desc;\\
  static const AsnChoiceMemberDesc      \\
					\>\>\_mdescs[];\\
\\
  const AsnTypeDesc             \>\>*\_getdesc() const;\\
  AsnType                       \>\>*\_getref (const char *membername, bool create = false);\\
\\
\<\#if TCL\\
  int                   \>\>TclGetDesc (Tcl\_DString *) const;\\
  int                   \>\>TclGetVal (Tcl\_Interp *) const;\\
  int                   \>\>TclSetVal (Tcl\_Interp *, const char *valstr);\\
\<\#endif // TCL\\
\<\#endif // META\\
\<\};
\end{Ccode}



\subsection{\label{C++-set-of-section}SET OF and SEQUENCE OF}

Each SET OF and SEQUENCE OF type produces its own list class, unlike
the C backend which uses a single generic list type for all lists.
This makes the C++ list routines type safe which allows the C++
compiler to detect more programmer errors.

C++ templates should be used to reduce the code duplication when they
become widespread and reliably implemented.  The duplicated list
handling methods may bloat the size of the generated code.

Any tagging errors are reported and abort the decoding process via
{\C longjmp}.

From the EX1 ASN.1 module the following list is produced:
\begin{Ccode}
class T1SeqOf: public AsnType\\
\{\\
protected:\+\\
    unsigned long int\>\>count;\\
    struct AsnListElmt\\
    \{\+\\
        struct AsnListElmt \>*next;\\
        struct AsnListElmt \>*prev;\\
        AsnInt             \>*elmt;\-\\
    \}                   \>\>*first, *curr, *last;\\
\\
\<public:\\
                         \>\>T1SeqOf() \{ count = 0; first = curr = last = NULL; \}\\
                         \>\>\~{}T1SeqOf();\\
    AsnType              \>\>*Clone() const;\\
\\
    void                 \>\>SetCurrElmt (unsigned long int index);\\
    unsigned long int    \>\>GetCurrElmtIndex();\\
    void                 \>\>SetCurrToFirst();\\
    void                 \>\>SetCurrToLast();\\
\\
    // reading member fcns\\
    int                  \>\>Count() const;\\
    AsnInt               \>\>*First() const;\\
    AsnInt               \>\>*Last() const;\\
    AsnInt               \>\>*Curr() const;\\
    AsnInt               \>\>*Next() const;\\
    AsnInt               \>\>*Prev() const;\\
\\
    // routines that move the curr elmt\\
    AsnInt               \>\>*GoNext();\\
    AsnInt               \>\>*GoPrev();\\
\\
    // write \& alloc fcns--returns new elmt\\
    AsnInt               \>\>*Append();  // add elmt to end of list\\
    AsnInt               \>\>*Prepend(); // add elmt to beginning of list\\
    AsnInt               \>\>*InsertBefore(); // insert elmt before current elmt\\
    AsnInt               \>\>*InsertAfter();  // insert elmt after current elmt\\
\\
    // write \& alloc \& copy--returns list after copying elmt\\
    T1SeqOf              \>\>\&AppendCopy (AsnInt \&elmt);  // add elmt to end of list\\
    T1SeqOf              \>\>\&PrependCopy (AsnInt \&elmt); // add elmt to beginning of list\\
    T1SeqOf              \>\>\&InsertBeforeAndCopy (AsnInt \&elmt); // insert elmt before current elmt\\
    T1SeqOf              \>\>\&InsertAfterAndCopy (AsnInt \&elmt); // insert elmt after current elmt\\
\\
    // removing the current elmt from the list\\
    void                 \>\>RemoveCurrFromList();\\
\\
    // encode and decode routines\\
    AsnLen               \>\>BEncContent (BUF\_TYPE b);\\
    void                 \>\>BDecContent (BUF\_TYPE b, AsnTag tag, AsnLen elmtLen,\\
                         \` AsnLen \&bytesDecoded, ENV\_TYPE env);\\
\\
    AsnLen               \>\>BEnc (BUF\_TYPE b);\\
    void                 \>\>BDec (BUF\_TYPE b, AsnLen \&bytesDecoded, ENV\_TYPE env);\\
\\
    int                  \>\>BEncPdu (BUF\_TYPE b, AsnLen \&bytesEncoded);\\
    int                  \>\>BDecPdu (BUF\_TYPE b, AsnLen \&bytesDecoded);\\
\\
    void                 \>\>Print (ostream \&os);\\
\\
\<\#if META\\
  static const AsnListTypeDesc  \>\>\_desc;\\
  const AsnTypeDesc             \>\>*\_getdesc() const;\\
  AsnType                       \>\>*\_getref (const char *index, bool create = false);\\
\\
\<\#if TCL\\
  int                   \>\>TclGetDesc (Tcl\_DString *) const;\\
  int                   \>\>TclGetVal (Tcl\_Interp *) const;\\
  int                   \>\>TclSetVal (Tcl\_Interp *, const char *valstr);\\
  int                   \>\>TclUnsetVal (Tcl\_Interp *, const char *valstr);\\
\<\#endif // TCL\\
\<\#endif // META\\
\<\};
\end{Ccode}

Each list is doubly linked to allow simple reverse traversal for
backwards encoding.  The {\C prev} pointer will be {\C NULL} for the
first element of the list and the {\C next} pointer will be {\C NULL} for
the last element of the list.

Each list maintains a pointer to the current element of the list.
Several routines are provided to manipulate the current item.  Since
there is only one current pointer, you may have to save and restore
the current pointer with the {\C GetCurrElmtIndex} and
{\C SetCurrElmt} methods if you call routines that deal with the
list while iterating through it.

In addition to the standard encode, decode and print methods, some
list utility routines are included in each list class.  They are
fairly simple and are described briefly here.

\begin{Ccode}
void \>\>\>SetCurrElmt (unsigned long int index);
\end{Ccode}
This sets the current pointer to the element with the given index.
Indexes start at zero, that is, the first element in the list has an
index of zero.  If the given index is greater than or equal to the
number of elements in the list, the current pointer is set to the last
element of the list.

\begin{Ccode}
unsigned long int \>\>\>GetCurrElmtIndex();
\end{Ccode}
This returns the index of the current element.  If the current pointer
is {\C NULL} (or does not reference an element of the list, which is an
error condition), the index returned will be greater than or equal to
the number of elements in the list (indexes start at zero so this is
an invalid index).


\begin{Ccode}
void \>\>\>SetCurrToFirst();
\end{Ccode}
This sets the current pointer to the first element of the list. If the
list is empty, it is set to {\C NULL}\@.
\begin{Ccode}
void \>\>\>SetCurrToLast();
\end{Ccode}
This sets the current pointer to the last element of the list. If the
list is empty, it is set to {\C NULL}\@.


\begin{Ccode}
int \>\>\>Count() const;
\end{Ccode}
This returns the number of elements in the list.

\begin{Ccode}
AsnInt \>\>\>*First() const;\\
AsnInt \>\>\>*Last() const;\\
AsnInt \>\>\>*Curr() const;\\
AsnInt \>\>\>*Next() const;\\
AsnInt \>\>\>*Prev() const;
\end{Ccode}
The above routines return a pointer to the list element that the
routine name indicates.  They return {\C NULL} if the requested element is
not present.  For example {\C First} will return a pointer to the
first element in the list or {\C NULL} if the list is empty.  These
routines do not affect the state of the list; the current pointer and
the count remain the same.

\begin{Ccode}
AsnInt \>\>\>*GoNext();\\
AsnInt \>\>\>*GoPrev();
\end{Ccode}
These routines change the current pointer to the next/previous element
and return a pointer to that element.  If the current element is {\C NULL} or
points to the last element, {\C GetNext} returns {\C NULL}\@. Similarly, if
the current element is {\C NULL} or points to the first element, {\C GetPrev}
returns {\C NULL}\@.


\begin{Ccode}
AsnInt \>\>\>*Append();
\end{Ccode}
This allocates a new list element, appends it to the end of the list
and returns a pointer to the new list element.  Notice that you must
set the value of the returned list element.

\begin{Ccode}
AsnInt \>\>\>*Prepend();
\end{Ccode}
This allocates a new list element, prepends it to the beginning of the
list and returns a pointer to the new list element. You must set the
value of the returned list element.

\begin{Ccode}
AsnInt \>\>\>*InsertBefore();
\end{Ccode}
This allocates a new list element, inserts it before the current list
element and returns a pointer to the new list element.  You must set
the value of the returned list element.  If the current pointer is
{\C NULL}, the new element is placed at the beginning of the list.

\begin{Ccode}
AsnInt \>\>\>*InsertAfter();
\end{Ccode}
This allocates a new list element, inserts it after the current list
element and returns a pointer to the new list element.  You must set
the value of the returned list element.  If the current pointer is
{\C NULL}, the new element is placed at the end of the list.

\begin{Ccode}
T1SeqOf \>\>\>\&AppendCopy (AsnInt \&elmt);\\
T1SeqOf \>\>\>\&PrependCopy (AsnInt \&elmt);\\
T1SeqOf \>\>\>\&InsertBeforeAndCopy (AsnInt \&elmt);\\
T1SeqOf \>\>\>\&InsertAfterAndCopy (AsnInt \&elmt);
\end{Ccode}
These are similar to the {\C Append}, {\C Prepend},
{\C InsertBefore} and {\C InsertAfter} routines except that a
copy of the given element's value is placed in the list and the list
itself is returned.

\subsection{\label{C++-enumerated-section}ENUMERATED, Named Numbers and Named Bits}

The C++ type generator encapsulates each ENUMERATED type, INTEGER
with named numbers and BIT STRING with named bits in a new class that
inherits from the proper base class and defines the named elements.
This provides a separate scope for these identifiers so their symbol
will be exactly the same as their ASN.1 counterpart.  Currently these
identifiers are not checked for conflicts with C++ keywords, so you
may have to modify some of them in the ASN.1 modules.

Inheritance is used for attaching ENUMERATED, named number and named
bit information.  ENUMERATED types inherit from the {\C AsnEnum} class,
INTEGERs with named number types inherit from the {\C AsnInt} class and BIT
STRINGs with named bits inherit from the {\C AsnBits} class.

If the tagging on the type is different from the type it inherits
from, the PDU encode and decode methods are re-defined with the
correct tags to override the PDU encode and decode methods of the base
class.

As with the other types, any tagging errors are reported and abort the
decoding process via {\C longjmp}.  No range checking is done on the
decoded values although it would be easy to provide a new
{\C BDecContent} method in the new class that calls the base class's
and then checks the range of the result.

\begin{Ccode}
/* ENUMERATED { a(0), b(1), c(2) } */\\
class T1Enum: public AsnEnum\\
\{\\
public:\\
\#if TCL\\
         \>               \>\>T1Enum(): AsnEnum (\_nmdescs[0].value) \{\}\\
\#else\\
         \>               \>\>T1Enum(): AsnEnum () \{\}\\
\#endif\+\\
                          \>\>T1Enum (int i): AsnEnum (i) \{\}\\
  enum\\
  \{\+\\
    a = 0,\\
    b = 1,\\
    c = 2\-\\
  \};\\
\\
\<\#if META\\
  static const AsnNameDesc      \>\>\_nmdescs[];\\
  static const AsnEnumTypeDesc  \>\>\_desc;\\
  const AsnTypeDesc             \>\>*\_getdesc() const;\\
\<\#endif // META\\
\<\};
\end{Ccode}

\section{\label{val-gen-C++-section}ASN.1 to C++ Value Translation}

C++ {\C const} values are used to hold ASN.1 defined values.  C++
values will be produced for INTEGER, BOOLEAN and OBJECT IDENTIFIER
ASN.1 values.  An {\C extern} declaration for each {\C const} value is
written at the end of the header file of the value's module.  The
{\C const} values are defined at the beginning of the {\ufn .C} file
of the value's module.  The {\C extern} declarations are at the end
of the header file so that any required class definitions are
available.

The following is from the end of the header file generated for the EX1
module:
\begin{Ccode}
extern const AsnOid  \>\>\>anOidVal;\\
extern const AsnOid  \>\>\>theSameOidVal;\\
extern const AsnInt  \>\>\>anIntVal;\\
extern const AsnBool \>\>\>aBoolVal;
% \\ extern const AsnInt  \>\>\>foobar;
\end{Ccode}

The following is from the beginning of the {\ufn .C} file generated
for the EX1 module:
\begin{Ccode}
const AsnOid  \>\>\>anOidVal (2, 40, 29);\\
const AsnOid  \>\>\>theSameOidVal (2, 40, 29);\\
const AsnInt  \>\>\>anIntVal (1);\\
const AsnBool \>\>\>aBoolVal (true);
% \\ const AsnInt  \>\>\>foobar (29);
\end{Ccode}

The C++ constructor mechanism is used to generate these values.  This
mechanism is superior to C static initialization because it allows C++
code to be run to initialize the values.

\section{\label{compiler-dir-C++-section}Compiler Directives}
Compiler directives are ignored by the C++ backend of snacc.  If you want
to implement them, look at the {\C FillCxxTypeDefInfo} routine in
file {\ufn \dots/compiler/back-ends/c++-gen/types.c}.  Then look at the
way it is done for the C backend (file
{\ufn \dots/compiler/back-ends/c-gen/type-info.c})

\section{\label{compiling-gen-C++-section}Compiling the Generated C++ Code}

When compiling the generated C++ code you will need:
\begin{enumerate}
\item
  The include directory where the files from {\ufn \dots/c++-lib/inc/} have been installed in your include path so that the C++ sources can include these library header files.
  The header files should be included with statements like {\C \#include <snacc/c++/asn-incl.h>} and your C++ compiler should be supplied with {\ufn -I/usr/local/include} in case snacc got installed under {\ufn /usr/local/}.
\item
  to link with the C++ ASN.1 runtime library, {\ufn \dots/c++-lib/libasn1c++.a}.
  In case snacc got installed under {\ufn /usr/local/}, your linker may need to be supplied with {\ufn -L/usr/local/lib} and {\ufn -lasn1c++} as arguments.
\item
  to link with the math library ({\ufn -lm}), since the ASN.1 REAL type's encode and decode routine use some math routines.
\end{enumerate}

See the example in {\ufn \dots/c++-examples/simple/} for a complete
example. The makefile and main routines are probably the most
important.  There are several other examples in the
{\ufn \dots/c++-examples/} directory.
