% file: .../doc/c-gen.tex

% $Header: /cvs/Darwin/src/live/Security/SecuritySNACCRuntime/doc/c-gen.tex,v 1.1.1.1 2001/05/18 23:14:10 mb Exp $
% $Log: c-gen.tex,v $
% Revision 1.1.1.1  2001/05/18 23:14:10  mb
% Move from private repository to open source repository
%
% Revision 1.1.1.1  1999/03/16 18:05:51  aram
% Originals from SMIME Free Library.
%
% Revision 1.1  1997/01/01 22:47:33  rj
% first check-in
%

\chapter{\label{c-code-gen-chapter}C Code Generation}
\section{\label{intro-C-section}Introduction}
Snacc was designed primarily to provide high-performance encoders and
decoders.  Key areas to optimize are buffer and memory management.
Buffers are used to hold encoded values and the memory management is
used when building the internal representation of a value when decoding.

C macros are used where possible to eliminate function call overhead
for small, commonly used routines.  Using macros with constant
expressions as parameters allows smarter C compilers to do some of the
calculations at compile time.  In general, short-cuts that can be taken
without sacrificing the robustness of code are used.

The generated code can be quite large; large reductions of the size of
the binaries can be achieved by using the optimizing options of your C
compiler.

We will use an example ASN.1 module, EX1, to help explain snacc's code
generation.  The EX1 module uses some of the common built-in types and
contains some simple values.  The field names have been left out
to show snacc naming conventions.  The C generation code is in
{\ufn \dots/compiler/back-ends/c-gen/} if you want to alter it.

\begin{small}
\begin{verbatim}
EX1 DEFINITIONS ::=
BEGIN

anOidVal      OBJECT IDENTIFIER ::= { joint-iso-ccitt 40 foobar(29) }
theSameOidVal OBJECT IDENTIFIER ::= { 2 40 29 }
anIntVal      INTEGER ::= 1
aBoolVal      BOOLEAN ::= TRUE

T1 ::= SEQUENCE
{
   INTEGER OPTIONAL,
   OCTET STRING OPTIONAL,
   ENUMERATED { a(0), b(1), c(2) },
   SEQUENCE OF INTEGER,
   SEQUENCE { id OBJECT IDENTIFIER, value OCTET STRING },
   CHOICE { INTEGER, OBJECT IDENTIFIER }
}

END
\end{verbatim}
\end{small}

\noindent
Use the following command to compile the EX1 ASN.1 module:

\noindent
{\ufn \%1 snacc -u \dots/asn1specs/asn-useful.asn1 \dots/asn1specs/ex1.asn1}
\noindent

This produces the files {\ufn ex1.h} and {\ufn ex1.c}.

For each ASN.1 type an equivalent C data type, a BER encoding routine,
a BER decoding routine, a printing routine and a freeing routine will
be generated.  C values will also be generated from simple ASN.1
values.  Each aspect of the C code generation will be discussed in the
next sections.

\section{\label{naming-C-section}ASN.1 to C Naming Conventions}

For any given module, snacc may produce C type definitions, functions
and {\C \#define}s.  We assume that all C {\C typedef},
{\C struct}, {\C enum} and {\C union} tag, {\C enum} value,
variable, {\C \#define} and function names share a single name space.

The C type name for a type is the same as its ASN.1 type name (with
any hyphens converted to underscores) unless there is a conflict.
Since, unlike ASN.1, the C types for each ASN.1 module share the same
name space, snacc makes sure the C typenames are unique amoung all the
modules and that they do not conflict with C keywords.  The conflicts
are resolved by appending digits to the conflicting name.  To avoid
confusing numbered type names etc., you should edit the ASN.1 source
and name them properly.

Named numbers, ENUMERATED values and named bits are put in entirely in
upper case to match the common C convention for {\C \#define} and
{\C enum} values.

Empty field names in SETs, SEQUENCEs, and CHOICEs will be filled.  The
field name is derived from the type name for that field. The library
types such as INTEGER have default field names defined by the compiler
(see {\ufn \dots/compiler/back-ends/c-gen/rules.c} and
{\ufn \dots/compiler/back-ends/c++-gen/rules.c}).  The first letter of the field
name is in lower case.  Again, empty field names should be fixed
properly by adding them to the ASN.1 source.

New type definitions will be generated for SETs, SEQUENCEs, CHOICEs,
ENUMERATED, INTEGERs with named numbers and BIT STRING with named bits
whose definitions are embedded in other SET, SEQUENCE, SET OF,
SEQUENCE OF, or CHOICE definitions.  The name of the new type is
derived from the name of the type in which it was embedded.  Perhaps a
better way would use the field name as well, if present.

\section{\label{type-gen-C-section}ASN.1 to C Data Structure Translation}

To handle the different scoping rules between ASN.1 and C, the names
of some ASN.1 data structure elements such as ENUMERATED type symbols
may be altered to avoid conflicts.  The T1 type in example ASN.1
module EX1 has no field names so snacc will generate them.  It is
recommended to provide field names in the ASN.1 source instead of
relying on compiler generated names.  The following is the generated C
data structure for the EX1 module from the {\ufn ex1.h} file (function
prototypes have been removed):

\begin{small}
\begin{verbatim}

typedef enum
{
    A = 0,
    B = 1,
    C = 2
} T1Enum; /* ENUMERATED { A(0), B(1), C(2) }  */

typedef struct T1Choice /* CHOICE */
{
    enum T1ChoiceChoiceId
    {
        T1CHOICE_INT1,
        T1CHOICE_OID
    } choiceId;
    union T1ChoiceChoiceUnion
    {
        AsnInt int1; /* INTEGER */
        AsnOid *oid; /* OBJECT IDENTIFIER */
    } a;
} T1Choice;

typedef struct T1Seq /* SEQUENCE */
{
    AsnOid id; /* OBJECT IDENTIFIER */
    AsnOcts value; /* OCTET STRING */
} T1Seq;

typedef AsnList T1SeqOf; /* SEQUENCE OF INTEGER */

typedef struct T1 /* SEQUENCE */
{
    AsnInt *int1; /* INTEGER OPTIONAL */
    AsnOcts octs; /* OCTET STRING OPTIONAL */
    T1Enum t1Enum; /* T1Enum */
    T1SeqOf *t1SeqOf; /* T1SeqOf */
    struct T1Seq *t1Seq; /* T1Seq */
    struct T1Choice *t1Choice; /* T1Choice */
} T1;
\end{verbatim}
\end{small}

Every ASN.1 type definition maps into a C {\C typedef}.  SETs and
SEQUENCEs map into C structures and other simple types map into their
obvious C counterpart.  SET OF and SEQUENCE OF types map into a
generic list type which is doubly linked and NULL terminated.  The
reverse link on the lists allows for simpler backwards encoding.  More
information on the library types can be found in Chapter~\ref{lib-C-chapter}.

Comments that contain a fragment of each type's ASN.1 definition are
inserted in the header file to clarify cases where elements have been
re-named.

Aggregate types that are defined in other type definitions are moved
to their own type definitions.  For example, notice how the SEQUENCE
and CHOICE that are in type {\C T1} have been moved to the types
{\C T1Seq} and {\C T1Choice} in the C code.  This simplifies code
generation at the cost of introducing new types.

Identifiers for named numbers from INTEGER and ENUMERATED types and
named bits from the BIT STRING type are capitalized in the C
representation.  The ENUMERATED type maps to a C {\C enum} and the INTEGER
and BIT STRING named numbers/bits are handled with {\C \#define}
statements.

Most OPTIONAL elements of SEQUENCEs and SETs are referenced by
pointer.  An element is considered present if its pointer is non-NULL\@.
OCTET STRINGs, BIT STRINGs and OBJECT IDENTIFIERs are the exceptions,
and are included by value even when they are OPTIONAL because they are
small and contain an internal pointer that can be used to determine
their presence.  For an example of this, look at the first two
elements of type {\C T1}.  The INTEGER type is referenced by pointer
because it is OPTIONAL, but the OCTET STRING type is included
(non-pointer) in the {\C T1} type even though it is OPTIONAL\@.


\section{\label{encode-gen-C-section}Encode Routines}

Snacc generates two kinds of encoding routines. One is PDU oriented
and encodes the type's tag, length and content and the other
only encodes the type's content.  The generated encoders only call the
content encoders, except in the case of ANY and ANY DEFINED BY types.
Typically, you will only call the PDU oriented routines from your
code.

The content and PDU encoding routine interfaces are similar for all
ASN.1 types.  They both take two parameters, one is a buffer pointer
and the other is a pointer to the value to be encoded.  For example
the {\C T1} type from module EX1 has the following prototypes for
its encoding routines.

\begin{verbatim}
AsnLen BEncT1Content (BUF_TYPE b, T1 *v);
AsnLen BEncT1 (BUF_TYPE b, T1 *v);
\end{verbatim}

{\C BEnc} is short for ``BER Encode''. The {\C BUF\_TYPE}
parameter is the buffer to encode the value into and the {\C T1~*}
parameter is a pointer to the instance of the {\C T1} type that is to be
encoded.

The {\C BEncT1Content} routine only encodes the content of a {\C T1}
type and returns its encoded length; it does not encode its tag
(UNIVERSAL (CONSTRUCTED) 16 for SEQUENCE) or length.  The job of
encoding the tag and length is up to any type that encapsulates {\C T1}.
This design allows decisions about implicit tagging to be made at code
generation time instead of runtime, improving performance.  Also,
different encoding rules may fit into this model more easily.

The {\C BEncT1} routine encodes the tag (UNIVERSAL (CONSTRUCTED) 16
for SEQUENCE), length and content of a {\C T1} type and returns its encoded
length.  This is the PDU oriented routine and will only be generated
if the user designates the type as a PDU type via a compiler directive
or the type is used as
the content of an ANY or ANY DEFINED BY type (as indicated by an
OBJECT-TYPE macro).  A PDU type is a type that defines an entire PDU;
the user will typically be calling the encode and decode routine for
PDU types directly.  See Section~\ref{compiler-dir-C-section} for how to
designate PDU types with compiler directives.

The snacc encoders are somewhat strange; they encode a value starting
from the end of its BER representation and work back to its beginning.
This ``backwards'' encoding technique simplifies the use of definite
lengths on constructed values.  Other encoders that encode forwards,
such as those of CASN1, use an intermediate buffer format so that a
buffer containing the encoded length of a constructed value can be
inserted before its encoded content, after the content has been
encoded.  Use of intermediate buffers hurts performance.  Other
compilers' approaches have been to only encode indefinite lengths for
constructed values, however, this will not support some encoding rules
such as DER\@.  The drawback of encoding backwards is that BER values
cannot be written to stream-oriented connections as they are encoded.

Both definite and indefinite length encodings for constructed values'
lengths are supported.  Currently the choice is made when compiling
the generated code, via the {\C USE\_INDEF\_LEN} flag.  If both length
forms, definite and indefinite, are required, it easy to modify the
length encoding macros to check a global variable for the length
form to use.  For most types, using definite lengths produces smaller
encodings with little performance difference.

After calling an encode routine you should always check the buffer
you encoded into for a write error.  This is the only error reporting
mechanism used for the encoders.  See the C buffer section (Section
\ref{lib-buf-section}) for how to check a buffer for a write error.


\section{\label{decode-gen-C-section}Decode Routines}

Decoding routines are like the encoding routines in that there are two
kinds, one that decodes the type's tag, length and content and one
that only decodes the type's content.  As mentioned in the encoder
section, the content style interface allows implicit tagging decisions
to be made at compile time.

Unlike the encoding routines, the PDU and content decoding routines
take different arguments.  For the {\C T1} type the following would be
produced:
\begin{verbatim}
void BDecT1Content (BUF_TYPE b, AsnTag tagId0, AsnLen elmtLen0, T1 *v, AsnLen *bytesDecoded, ENV_TYPE env);
void BDecT1 (BUF_TYPE b, T1 *v, AsnLen *bytesDecoded, ENV_TYPE env);
\end{verbatim}

Notice that the content decoder, {\C BDecT1Content}, has tag and
length parameters that the PDU decoder, {\C BDecT1}, does not have.
Since the content decoder does not decode the tag and length on the
value, it is necessary to pass them in as parameters. Only OCTET
STRING and BIT STRING decoders will actually use the information
in the tag parameter.

The {\C BUF\_TYPE} parameter is the buffer that holds the BER value
being decoded.

The {\C tagId0} parameter is the last tag that was decoded on the
content of the type that is about to be decoded.  In the case of type
{\C T1}, {\C BDecT1Content} gets a tagId0 of UNIVERSAL (CONSTRUCTED) 16,
unless it is implicitly tagged by another type.  Most content decoding
routines ignore the tag information. OCTET STRING and BIT STRING
decoders use the tag information to determine whether the contents are
constructed or primitive.  CHOICE decoders use the tag information to
determine which CHOICE element is present.  CHOICE values are treated
differently, as will be explained shortly.

The {\C elmtLen0} parameter is the length of the content of the type
being decoded.  This is simply the length decoded from the buffer by
the containing type's decoder just before calling this decode routine.

The {\C v} parameter is a pointer to space allocated for the type
being decoded.  This memory is not allocated by the decoding routine
itself; this supports the cases where the type is enclosed in the
struct of the parent (i.\,e.\  no extra allocation is necessary).  If
the type to be decoded is referenced by pointer from its parent type,
the parent type's decoding routine must allocate the type.

The {\C bytesDecoded} parameter maintains the running total of the
number of octets that have been decoded.  For example, if I call
{\C BDecT1Content} with a {\C bytesDecoded} parameter that points
to 20 and the encoded length of the {\C T1} value is 30 octets,
{\C bytesDecoded} will point to 50 when {\C BDecT1Content}
returns.  Maintaining the length is vital to determining the presence
or absence of OPTIONAL elements in a SET or at the end of SEQUENCE\@.
Local variables are used to hold the lengths; there is no global stack
of lengths as with CASN1.

The {\C env} parameter is used in conjunction with {\C longjmp}
calls.  When an decoder encounters a fatal error such as a missing
tag, it uses the {\C env} with a {\C longjmp} call to pop back to the
initial decode call. Section~\ref{lib-err-C-section} has more error
management details.

CHOICEs are decoded a little differently from other types.  For all
types except CHOICEs, all of the tag and length pairs on the content
are decoded by the parent type, and the last pair a passed into to
content decoding routine via the {\C tagId0} and {\C elmtLen0}
parameters.  For CHOICEs, all of the tag and length pairs on the
content are decoded and then the first tag and length pair in the
CHOICE content is decoded by the parent and passed into the CHOICE
content decoding routine.  The first tag in a CHOICE's content is the
important tag by which the CHOICE determines which element is present.
This technique simplifies the code for dealing with untagged CHOICEs
embedded in other CHOICEs.  CHOICEs nested in this way mean that a
single tag determines which element is present in more than one
CHOICE\@.

The decoding routines allocate memory to hold the decoded value.  By
default snacc decoders use nibble memory (see Section
\ref{lib-mem-C-section}) which is very efficient in allocation and
virtually cost free for freeing.

To save memory, decoders generated by some other tools build values
that reference the data in the encoded PDU for types like OCTET
STRING\@.  Snacc decoded values do not reference the BER data in any way
for several reasons.  One, the encoded value may be held in some
bizarre buffer making access to the value difficult.  Two, with more
encoding rules being formalized, this technique may not always work
since the encoded format may be different from the desired internal
format.  Three, snacc decoders concatenate any constructed BIT and
OCTET STRINGs values when decoding, to simplify processing in the
application.

Snacc decoders can detect a variety of errors which will be reported
by {\C longjmp}.  Any tagging errors are reported. SETs must contain
all non-OPTIONAL components and SEQUENCEs must be in order and contain
all non-OPTIONAL components. Extra components in SETs and SEQUENCEs
are considered an error.  Errors will also be reported if you attempt
to decode values that exceed the limitations of the internal
representation (e.\,g.\ an integer that is larger than a
{\C long int} allows).

\section{\label{print-gen-C-section}Print Routines}
All of the generated print routines take similar parameters. For
example the {\C T1} type's print routine prototype is:
\begin{verbatim}
void PrintT1 (FILE *f, T1 *v, unsigned short int indent);
\end{verbatim}

The print routine writes the given value, {\C v}, to the given {\C FILE~*},
{\C f}.  The printed value is indented by {\C indent} spaces.  The
values are printed in an ASN.1 value notation style.  {\C PrintT1}
prints in the following style:
\begin{small}
\begin{verbatim}
{ -- SEQUENCE --
    17,
    '436c696d6220617420537175616d697368'H  -- "Climb at Squamish" --,
    0,
    { -- SEQUENCE OF --
        18,
        19
    },
    { -- SEQUENCE --
        id {2 40 29},
        value '736f6d6520737472696e67'H  -- "some string" --
    },
    20
}
\end{verbatim}
\end{small}

OCTET STRINGs are printed in a hexadecimal notation, and any printable
characters are included after the string in an ASN.1 comment.  Note
that the enumerated type value, 0, did not print its symbol, ``A''
from the ENUMERATED type.  It would be fairly easy to modify the C and
C++ back ends to generate print routines that printed the ENUMERATED
types' symbols instead of their values.

\section{\label{free-gen-C-section}Free Routines}

Snacc generates free routines of the form:
\begin{verbatim}
void FreeT1 (T1 *v);
\end{verbatim}

These routines will free all the components named type.
For example the above {\C FreeT1} routine will free all the
components of the given {\C T1} value, but not the {\C T1} value itself.  The
passed in pointer is not freed because it may be embedded in another
type which will be freed by another call to {\C Asn1Free}.  All the pieces
of memory are freed using the {\C Asn1Free} macro defined in
{\ufn asn-config.h}.  Each library type has its own free routine that
may call {\C Asn1Free}. The values are typically allocated during
decoding, using the {\C Asn1Alloc} macro.

The memory management can be changed by editing the {\ufn asn-config.h}
file to use you own memory management routines.  By default the memory
manager uses the nibble memory system described in Section
\ref{lib-mem-C-section}.  The nibble memory system does not need explicit
frees of each component so the generated free routines are not needed.
However, if you change the memory management to use something like
{\C malloc} and {\C free}, you should use the generated free routines.


\section{\label{val-gen-C-section}ASN.1 to C Value Translation}

C values will be produced for INTEGER, BOOLEAN and OBJECT IDENTIFIER
values.  C {\C extern} declarations for the value are put at the end
of the header file (after all of the type definitions).  The value
definitions are put at the beginning of the source file. For example,
the following will be produced for the EX1 module (at the end of
file ex1.h):

\begin{small}
\begin{verbatim}
extern AsnOid anOidVal;
extern AsnOid theSameOidVal;
extern AsnInt anIntVal;
extern AsnBool aBoolVal;
extern AsnInt foobar;
\end{verbatim}
\end{small}

(at the beginning of file ex1.c):

\begin{small}
\begin{verbatim}
AsnOid anOidVal = { 2, "\170\35" };
AsnOid theSameOidVal = { 2, "\170\35" };
AsnInt anIntVal = 1;
AsnBool aBoolVal = TRUE;
AsnInt foobar = 29;
\end{verbatim}
\end{small}

\section{\label{compiler-dir-C-section}Compiler Directives}

Snacc allows the user to control some aspects of the generated code by
inserting special comments in the ASN.1 source.  Warning! only the
{\ASN isPdu} directive has been tested to any extent. Use the others
very carefully and only if you really need to.  The compiler
directives have the form:

\begin{verbatim}
--snacc <attribute>:"<value>"  <attribute>:"<value>" ...
\end{verbatim}

The {\ASN attribute} is the name of one of the accepted attributes and
the {\ASN value} is what the {\ASN attribute}'s new value will be.
The attribute value pairs can be listed in a single {\ASN --snacc}
comment or spread out in a list of consecutive comments.

Compiler directives are only accepted in certain places in the ASN.1
code.  Depending on their location in the ASN.1 source, the compiler
directives affect type definitions or type references.  The directives
for type definitions and references are different.  Module level
compiler directives to specify output file names and other information
would be useful, but are not implemented.

Here is an example to present some of the compiler directives and
their uses.  Let's say your data structure always deals with
{\C PrintableStrings} that are null terminated (internally, not in
the encoding).  The default snacc string type is a structure that
includes a length and {\C char~*} for the string octets.  To change
the default type to a simple {\C char~*} the best way would be define
your own string type, let's say {\ASN MyString} as follows:

\begin{small}
\begin{verbatim}
Foo ::= SET
{
    s1 [0] MyString OPTIONAL,
    s2 [1] MyString,
    i1 [2] INTEGER
}

Bar ::= CHOICE
{
    s1 MyString,
    i1 INTEGER
}

Bell ::= MyString

MyString ::= --snacc isPtrForTypeDef:"FALSE"
             --snacc isPtrForTypeRef:"FALSE"
             --snacc isPtrInChoice:"FALSE"
             --snacc isPtrForOpt:"FALSE"
             --snacc optTestRoutineName:"MYSTRING_NON_NULL"
             --snacc genPrintRoutine:"FALSE"
             --snacc genEncodeRoutine:"FALSE"
             --snacc genDecodeRoutine:"FALSE"
             --snacc genFreeRoutine:"FALSE"
             --snacc printRoutineName:"printMyString"
             --snacc encodeRoutineName:"EncMyString"
             --snacc decodeRoutineName:"DecMyString"
             --snacc freeRoutineName:"FreeMyString"
             PrintableString --snacc cTypeName:"char *"
\end{verbatim}
\end{small}

All but the last {\ASN --snacc} comment bind with the {\ASN MyString} type
definition. The last directive comment binds with the {\ASN PrintableString}
type.  The C data structure resulting from the above ASN.1 and compiler
directives is the following:
\begin{small}
\begin{verbatim}
typedef char *MyString; /* PrintableString */

typedef struct Foo /* SET */
{
    MyString s1; /* [0] MyString OPTIONAL */
    MyString s2; /* [1] MyString */
    AsnInt i1; /* [2] INTEGER */
} Foo;

typedef struct Bar /* CHOICE */
{
    enum BarChoiceId
    {
        BAR_S1,
        BAR_I1
    } choiceId;
    union BarChoiceUnion
    {
        MyString s1; /* MyString */
        AsnInt i1; /* INTEGER */
    } a;
} Bar;

typedef MyString Bell; /* MyString */
\end{verbatim}
\end{small}

The compiler directives used on the {\ASN MyString} type have some
interesting effects.  Notice that {\ASN MyString} is not referenced by
pointer in the CHOICE, SET, or type definition, {\ASN Bell}.

The generated code for encoding field {\C s1} of {\C Foo} type
will use the code\linebreak``{\C MYSTRING\_NON\_NULL (\&fooVal-->s1)}'' to check
for the presence of the OPTIONAL {\C s1} field.  The code associated
with MYSTRING\_NON\_NULL should return TRUE if the {\C s1} field
value is present and might look like:
\begin{verbatim}
#define MYSTRING_NON_NULL(s) (*s != NULL)
\end{verbatim}

The argument to {\C optTestRoutine} routine will be a pointer to the
field type's defining type.  Note that in the above example,
{\ASN MyString} is a {\C char~*}, therefore the {\C MYSTRING\_NON\_NULL}
macro's argument will be a {\C char~**}.

Setting the {\ASN genPrintRoutine} etc. attributes to false makes
snacc not define or generate any encode, decode, print, or free
routines for the {\ASN MyString} type.  You must provide these
yourself; the best approach is to take the normal {\ASN PrintableString}
routines and modify them to handle your special string type.

The names of the encode, decode, print and free routines used for the
{\ASN MyString} type will be based on the ones given with the
{\ASN printRoutineName} etc.\  attributes.  Snacc will prepend a
``B'' (for BER) and append a ``Content'' to the encode and decode
routines names, so you must provide the {\C BEncMyStringContent} and
{\C BDecMyStringContent} routines.  You may also need the
{\C BEncMyString} and {\C BDecMyString} routines if {\ASN MyString} is a
PDU type or used in an ANY or ANY DEFINED type.

The {\ASN PrintableString} type has its C type name changed to
{\C char~*} by the last compiler directive.  Thus {\ASN MyString} is defined
as a {\C char~*}.  This directive applies to the {\ASN PrintableString}
type reference.  Note that these directives do not affect the tags or
the encoded representation of the {\ASN MyString} type

The location of the {\ASN --snacc} comment(s) is important.
{\ASN --snacc} comment(s) between the {\ASN ::=} sign and the
following type are associated with the type being defined. Any
compiler directives after the type and before the next type or value
definition are associated with the type.  Fields in SETs, SEQUENCEs
and CHOICEs can be modified by putting the compiler directive after
the comma that follows the field type that you wish to modify.  In the
case of the last element of one of these types, where there is no
comma, just place it after the field and before the closing bracket of
the parent type.

Attributes shadow the type attributes filled in during the target
language type information generation pass of the compiler.  The type
definition attributes are:

\begin{description}
\item[cTypeName] { this is the type name that the generated type will
have. Its value can be any string that is valid as a C type name.}

\item[isPdu]     { whether this is a PDU type.  A PDU type will have
extra interfaces to the encode and decode routines generated. Its
value can be ``TRUE'' or ``FALSE''}

\item[isPtrForTypeDef] { TRUE if other types defined solely by this type
definition are defined as a pointer to this type. Its
value can be ``TRUE'' or ``FALSE''.}

\item[isPtrForTypeRef]{ TRUE if type references to this type
definition from a SET or SEQUENCE are by pointer. Its
value can be ``TRUE'' or ``FALSE''.}

\item[isPtrInChoice] {TRUE if type references to this type definition
from a CHOICE are by pointer. Its value can be ``TRUE'' or ``FALSE''.}


\item[isPtrForOpt]  { TRUE if OPTIONAL type references to this type
definition from a SET or SEQUENCE are by pointer. Its value can be
``TRUE'' or ``FALSE''.}

\item[optTestRoutineName] {name of the routine to test whether an
OPTIONAL element of this type in a SET or SEQUENCE is present. The
routine should return TRUE if the element is present.  The value of
this field is usually just the name of a C macro that tests for NON-NULL\@.
The argument to the routine will be a pointer to the type definition's
type.  The optTestRoutineName value can be any string value.}

\item[defaultFieldName] { if this type is used in a SET, SEQUENCE or
CHOICE without a field name then this value is used with a digit
appended to it.  Its value can be any string that is a valid C field
name in a struct or union.}
\item[printRoutineName] { name of this type definition's printing
routine.  Its value can be any string that is a C function or
macro name.}
\item[encodeRoutineName]{ name of this type definition's encoding
routine. Its value can be any string that is a C function or
macro name.}
\item[decodeRoutineName]{ name of this type definition's decoding
routine. Its value can be any string that is a C function or
macro name.}
\item[freeRoutineName]  { name of this type definition's freeing
routine. Its value can be any string that is a C function or
macro name.}

\item[isEncDec] {If this type is used in a SET or SEQUENCE then it is not
encoded or decoded.  Its value can be ``TRUE'' or ``FALSE''.  This is
handy for adding your own types to a standard that are only for local
use, and are not included in encoded values.}

\item[genTypeDef] { TRUE if you want a C type to be generated for this
type definition. Its values can be ``TRUE'' or ``FALSE''.}

\item[genPrintRoutine] { TRUE if you want a printing routine to be
generated for this type definition. Its values can be ``TRUE'' or
``FALSE''.}
\item[genEncodeRoutine] { TRUE if you want an encoding routine to be
generated for this type definition. Its values can be ``TRUE'' or
``FALSE''.}
\item[genDecodeRoutine] { TRUE if you want a decoding routine to be
generated for this type definition. Its values can be ``TRUE'' or
``FALSE''.}
\item[genFreeRoutine] { TRUE if you want a free routine to be
generated for this type definition. Its values can be ``TRUE'' or
``FALSE''.}
\end{description}


The type reference attributes are slightly different from the type
definition attributes due to the semantic differences between a type
definition and a type reference.  Type references will inherit some of
their attributes from the referenced type definition.  The following
are the valid type reference attributes:
\begin{description}
\item[cTypeName] { this is the type name that the generated type will
have. Its value can be any string that is valid as a C type name.}

\item[cFieldName] { if this is a field in a CHOICE, SET or SEQUENCE
then this holds the C field name for this reference. Its value can be
any string that is valid as a C field name.}

\item[isPtr] { TRUE if this is a pointer to the type named by
cTypeName.  This is usually determined from the referenced type
definitions attributes. Its value can be ``TRUE'' or ``FALSE''.}

\item[optTestRoutineName] {if this field is an OPTIONAL component then
this is the name of the routine to test whether it is present.  The
routine should return TRUE if the element is present.  The value of
this is usually just the name of a C macro that tests for NULL\@. The
argument to the routine will be a pointer to the type definition's
type.  The optTestRoutineName value can be any string value.}

\item[printRoutineName] { name of this type reference's printing
routine.  This and the other routine name attributes are useful for
special instances of the referenced type.  It is easier to modify the
referenced type definition if you want every instance of this type to
use a certain print etc.\  routine.  Its value can be any string that is
a value C function or macro name.}

\item[encodeRoutineName]{ name of this type reference's encoding
routine. Its value can be any string that is a function or
macro name.}

\item[decodeRoutineName]{ name of this type reference's decoding
routine. Its value can be any string that is a C function or
macro name.}

\item[freeRoutineName]  { name of this type reference's freeing
routine. Its value can be any string that is a C function or
macro name.}

\item[isEncDec] { If this type is used in a SET or SEQUENCE then the
field is not encoded or decoded.  Its value can be ``TRUE'' or
``FALSE''.  This is handy for adding your own types to a standard that
are only for local use, and are not included in encoded values.}

\item[choiceIdSymbol] {if this is a component of a CHOICE, this string
attribute will be the defined/enum symbol whose value in the choiceId
field indicates the presence of this field.}
\item[choiceIdValue]  {if this is a component of a CHOICE, this integer
attribute will be the value associated with the symbol in choiceIdSymbol.}

\end{description}




\section{\label{compiling-gen-C-section}Compiling the Generated C Code}

The generated C code (and libraries) can be compiled by both ANSI and K\&R C compilers.
C function prototypes use the {\C PROTO} macro and C function declarations use the {\C PARAMS} macro.
These macros are defined in {\ufn \dots/snacc.h} and their definitions depend on whether the {\C \_\_USE\_ANSI\_C\_\_} flag has been defined in {\ufn \dots/config.h}.

When compiling the generated C code you will need:
\begin{enumerate}
\item
  The include directory where the files from {\ufn \dots/c-lib/inc/} have been installed into in your include path so the C sources can include the library header files.
  The header files should be included with statements like {\C \#include <snacc/c/asn-incl.h>} and your C compiler should be supplied with {\ufn -I/usr/local/include} in case snacc got installed under {\ufn /usr/local/}.
\item
  to link with the correct C ASN.1 runtime library, depending on the buffer type you choose.
  In case snacc got installed under {\ufn /usr/local/}, your linker may need to be supplied with {\ufn -L/usr/local/lib} and one of {\ufn -lasn1cebuf}, {\ufn -lasn1cmbuf} or {\ufn -lasn1csbuf} as arguments.
\item
  to link with the math library ({\ufn -lm}), since the ASN.1 REAL type's encode and decode routine use some math routines.
\end{enumerate}

See the example in {\ufn \dots/c-examples/simple/} for a complete
example. The makefile and main routines are probably the most
important.  There are several other examples in the
{\ufn \dots/c-examples/} directory.
