% file: .../doc/intro-1.1.tex

% $Header: /cvs/Darwin/Security/SecuritySNACCRuntime/doc/intro-1.1.tex,v 1.1.1.1 2001/05/18 23:14:10 mb Exp $
% $Log: intro-1.1.tex,v $
% Revision 1.1.1.1  2001/05/18 23:14:10  mb
% Move from private repository to open source repository
%
% Revision 1.1.1.1  1999/03/16 18:05:53  aram
% Originals from SMIME Free Library.
%
% Revision 1.1  1997/01/01 22:47:28  rj
% first check-in
%

\chapter{\label{intro-1.1}Introduction to Snacc Release~1.1}

Snacc compiles ASN.1 \cite{X.208} (Abstract Syntax Notation One)
modules into C, C++ or type tables.  The generated C or C++ code
contains equivalent data structures and routines to convert values
between the internal (C or C++) representation and the corresponding
BER \cite{X.209} (Basic Encoding Rules) format.  The name ``snacc'' is
an acronym for ``Sample Neufeld ASN.1 to C/C++ Compiler''.

This compiler was written so I could do some encoding performance
research for my M.Sc.  See \cite{Sample93-1}, or \cite{Sample93-2} for
the results of that research.  A techreport will soo be available
via ftp from UBC, in the same directory as snacc.

The ASN.1 data structure language can specify complex types such as
lists and recursively defined types. BER data values are defined
independently of any computer architecture, providing a universal data
value representation that is useful for sharing data in heterogeneous
networks.

The process of converting an ASN.1 value from its C or C++
representation into an equivalent BER data value is called encoding
and the reverse process is called decoding.  This document was written
assuming that the reader is familiar ASN.1 and BER.  Further
information on ASN.1 and BER can be found in \cite{ASN.1Book},
\cite{ASN.1Overview}, \cite{X.208} and \cite{X.209}.

Compiling ASN.1 into C is not a new idea but many other tools such as
UBC's CASN1 \cite{CASN1}, ISODE's PEPY/POSY \cite{ISODE}, and
commercial tools either do not parse ASN.1 '90, produce slow encoders
and decoders or are outrageously expensive. The aim of this tool is to
provide an ASN.1 compiler that parses ASN.1 '90, produces efficient
encoding and decoding routines and is freely available.  Effort has
been made to make the generated encoders and decoders relatively easy
to fit into different software environments.

The table driven encoders are useful for certain applications such as
protocol testing.  They are also useful if you need to dynamically
load new ASN.1 definitions.  It is also fairly simple to write your
own special ASN.1 tools based on tables (e.g. a protocol tester that
verifies that values conform to a given ASN.1 type definition).  The
price of the flexibility is speed; they are slower (~4 times) than the
compiled C and C++ versions.

Some of snacc's features include:
\begin {itemize}
\item {parses CCITT ASN.1 '90 including  subtype notation}
\item {can compile and link inter-dependent ASN.1 modules (IMPORTS/EXPORTS)}
\item {some X.400 and SNMP macros are parsed}

\item {macro {\em definitions} do not generate syntax errors but are
are not processed. The macro definitions are retained as a string
internally (if you want to modify the compiler to process them).}

\item {value notation is parsed. OBJECT IDENTIFIERs, INTEGERs
and BOOLEANs are translated to C/C++ values. Any other value in \{\}'s
is kept as a string internally (if you want to modify the compiler to
process them).}

\item {optionally supports ``;'' separated type or value definitions
      in the ASN.1 source. This is useful for dealing with some macros and
      other language ambiguities that introduce parsing problems.}
\item {ANY DEFINED BY types are supported via the SNMP OBJECT-TYPE macro}
\end{itemize}

\section{\label{old-install-section}Installing snacc}

First of all, if you haven't already done so, un-archive snacc to
produce the directory {\em snacc} and its contents.  The following tools
are required to compile snacc:
\begin{itemize}
\item {\verb$lex$ or GNU's \verb$flex$ (\verb$flex$ is recommended)}
\item {\verb$yacc$ or GNU's \verb$bison$ (\verb$bison$ is recommended)}
\item {a C compiler and \verb$make$}
\end{itemize}

Some versions of \verb$yacc$ will choke due to the large size of the
\verb$asn1.yacc$ file, however, I have had no problems with \verb$bison$.
Our \verb$yacc$ grammar for ASN.1 has 61 shift/reduce errors and 2
reduce/reduce errors.  Most of these errors were introduced when
certain macros were added to the compiler.  Some of the shift/reduce
errors will require you to follow the offending macro in the ASN.1
module with a semi-colon.  The reduce/reduce errors were introduced by
macros that have ``Type or Value Lists'' because the NULL Type and
NULL values use the same symbol, ``NULL''.  This is not a problem
since no real processing is done with the macros in question at the
present.

\verb$Lex$ will work for the \verb$asn1.lex$ file but \verb$flex$ will typically
produce a smaller executable.  Most versions of \verb$lex$ have a small
maximum token size that will cause problems for long tokens in the
ASN.1 source files, such as quoted strings.  To avoid this problem,
increase the \verb$YYLMAX$ value in the generated {\em lex.yy.c} file to at least
2048.  \verb$Flex$ does not seem to have this problem.

The compiler and library C code has been written to support ANSI or
non-ANSI C\@.  ANSI C is used by default; this can be configured in
\verb$snacc/c_include/asn_config.h$.

By default, the compiler's makefiles use \verb$flex$, \verb$bison$ and
\verb$gcc$.  If you wish to change these, edit the following files:

\begin{verbatim}
snacc/src/makefile
snacc/src/c_lib/makefile
snacc/src/back_ends/c_gen/makefile
snacc/src/back_ends/c++_gen/makefile
\end{verbatim}

The C runtime library uses \verb$gcc$, and its makefile is
\verb$snacc/c_lib/makefile$.  The C++ runtime library uses \verb$g++$
(\verb$gcc-2.2.3$) and its makefile is \verb$snacc/c++_lib/makefile$.
The type table library makefile uses \verb$gcc$ and is
\verb$snacc/tbl_lib/makefile$.

Finally, to compile \verb$snacc$ and the C and C++ runtime libraries,
type the following at the shell prompt:

\begin{verbatim}
%1 cd snacc
%2 make all
\end{verbatim}

If you wish to install only the C (including type tables) or only the
C++ versions of the library, type \verb$make c$ or \verb$make c++$,
respectively, instead of \verb$make all$.  If the make succeeds, the
snacc binary, {\em snacc}, should be in the \verb$snacc/bin/$
directory, the C runtime libraries, {\em libasn1csbuf.a}, {\em
libasn1cebuf.a}, and {\em libasn1cmbuf.a}, should be in the
\verb$snacc/c_lib$ and the C++ runtime library, {\em libasn1c++.a}
should be in the \verb$snacc/c++_lib$.  The type table library, {\em
libasn1tbl.a} will be in \verb$snacc/tbl_lib$.  The type table tools,
{\em ptbl}, {\em pval} and {\em mkchdr} will be in \verb$snacc/bin$.
The \verb$.o$ and other junk files will have been removed.

After compiling the libraries, you can test the library routines with
\linebreak \verb$snacc/c_examples/test_lib/test_lib$ or \linebreak
\verb$snacc/c++_examples/test_lib/test_lib$.  These programs run
simple encoding and decoding tests on all of the library types.  You
can test the snacc compiler with the other examples.

A manual page that contains information on running snacc can be found in
\verb$snacc/doc/snacc.1$.  This should be installed in section~1 of
the manual.  You can use \verb$nroff -man snacc.1$ to view it if you
don't want to install it.


\section{\label{old-run-section}Running snacc}

Snacc is typically invoked from the shell command line and has the synopsis:
\begin{verbatim}
snacc [-h] [-P] [-t] [-e] [-d] [-p] [-f]
      [ -c | -C | -T <table output file>]
      [-u <useful types ASN.1 file>]
      [-mf <max file name len>]
      [-l <neg number>]
      <ASN.1 file list>
\end{verbatim}

Snacc generates C or C++ source code for BER encode and decode
routines as well as print and free routines for each type in the given
ASN.1 modules.  Alternatively, snacc can produce type tables that can
be used for table based/interpreted encoding and decoding.  The type
table based methods tend to be slower than their C or C++ counterparts
but they tend use less memory (table size vs. C/C++ object code).

Most of the 1990~ASN.1 features are parsed although some do not affect
the generated code.  Fairly rigourous error checking is performed on
the ASN.1 source; any errors detected will be reported (printed to
\verb$stderr$).

Each file in the ASN.1 file list should contain a complete ASN.1
module.  ASN.1 modules that use the IMPORTS feature must be compiled
together (specify all necessary modules in the ASN.1 file list).  The
generated source files will include each module's header file in the
command line order.  This makes it important to order the modules from
least dependent to most dependent on the command line to avoid type
ordering problems. Currently, snacc assumes that each ASN.1 file
given on the command line depends on all of the others on the command
line.  No attempt is made to only include the header files from
modules referenced in the import list for that module.

If the target language is C, snacc will generate a \verb$.h$ and
\verb$.c$ file for each specified ASN.1 module.  If the target
language is C++, snacc will generate a \verb$.h$ and \verb$.C$ file
for each module.  The generated file names will be derived from the
module names.

The command line options are:
% zap bullet for items
%\def\labelitemi{}

\begin{description}
\item[-h    ] {short for ``help'', prints a synopsis of snacc
and exits.}

\item[-c    ] {causes snacc to generate C source code.
This is the default behaviour of snacc if neither of the \verb$-c$ or
\verb$-C$ options are given.  Only one of the \verb$-c$, \verb$-C$ or
\verb$-T$ options should be specified.}

\item[-C    ] {causes snacc to generate C++ source code.}

\item[-T {\em file}] {causes snacc to generate type tables and
write them to the given file {\em file}.}

\item[-P    ] {causes snacc to print the parsed ASN.1
modules to \verb$stdout$ after the types have been linked, sorted, and
processed.  This option is useful for debugging snacc and observing
the modifications snacc performs on the types to make code generation
simpler.}
\end{description}

The options, \verb$-t, -v, -e, -d, -p,$ and \verb$-f$ affect
what types and routines go into the generated source code.
These options do not affect type table generation. If none of
them are given on the command line, snacc assumes that all of them are
in effect.  For example, if you do not need the Free or Print
routines, you should give the \verb$-t -v -e -d$ options to snacc.
This lets you trim the size of the generated code by removing
unnecessary routines; the code generated from large ASN.1
specifications can produce very large binaries.

\begin{description}
\item[-t    ] {causes snacc to generate type definitions in the
target language for each ASN.1 type.}

\item[-v    ] {causes snacc to generate value definitions in the
target language for each ASN.1 value.  Currently value definitions are
limited to INTEGERs, BOOLEANs and OBJECT IDENTIFIERs.}

\item[-e    ] {causes snacc to generate encode routines in the
target language for each ASN.1 type.}

\item[-d    ] {causes snacc to generate decode routines in the
target language for each ASN.1 type.}

\item[-p    ] {causes snacc to generate print routines in the
target language for each ASN.1 type.}

\item[-f    ] {causes snacc to generate free routines in the
target language for each ASN.1 type.  This option only works when the
target language is C\@.  The free routines hierarchically free C values.
A more efficient approach is to use the provided nibble-memory system.
The nibble memory permits freeing an entire decoded value without
traversing the decoded value.  This is the default memory allocator
used by snacc generated decoders.  See file
\verb$snacc/c_include/asn_config.h$ to change the default memory
system.  For more information on the memory management see Section~\ref{lib-mem-C-section}.}

\item[-u {\em file}] {causes snacc to read the useful types
definitions from the ASN.1 module in file {\em file}
for linking purposes.  For some ASN.1 specifications, such as SNMP,
the useful types are not needed.  The types in the given useful types
file are globally available to all modules; a useful type definition
is overridden by a local or explicitly imported type with the same
name. The useful type module can be found in
\verb$snacc/asn1specs/asn-useful.asn1$ and contains:


\begin{itemize}
\setlength{\itemsep}{0pt}
\setlength{\parsep}{0pt}
\nspace{0}
\item ObjectDesccriptor
\item NumericString
\item PrintableString
\item TeletexString
\item T61String
\item VideoTexString
\item IA5String
\item GraphicString
\item ISO646String
\item GeneralString
\item UTCTime
\item GeneralizedTime
\item EXTERNAL
\end{itemize}}


\item[-mf {\em number}] {causes the names of the generated source
files to have a maximum length of {\em number} characters, including
their suffix.  The {\em number} argument must be at least~3.  This option
is useful for supporting operating systems that only support short
file names.  A better solution is to shorten the module name of each
ASN.1 module.}

\item[-l {\em number}] {this is fairly obscure but may be useful.  Each
error that the decoders can report is given an id number.  The number
{\em number} is where the error ids start decreasing from as they are
assigned to errors.  The default is -100 if this option is not given.
Avoid using a number in the range -100 to 0 since they may conflict
with the library routines' error ids.  If you are re-compiling the
useful types for the library use -50.  Another use of this option is
to integrate newly generated code with older code; if done correctly,
the error ids will not conflict.}

\end{description}

Since ASN.1 has different scoping rules than C and C++, some name munging
is done for types, named-numbers etc. to eliminate conflicts.
Some capitalization schemes were chosen to fit common C programming
style.  For all names, dashes in the ASN.1 source are converted to
underscores.  See Sections \ref{naming-C-section} and \ref{naming-C++-section}
for more naming information.

The module name is used as a base name for the generated source file
names.  It will be put into lowercase and dashes will be replaced with
underscores.  Module names that result in file names longer than
specified with the \verb$-mf$ option will be truncated.  If the
\verb$-mf$ option was not given, file names will be truncated if they
are too long for the target file system. You may want to shorten long
module names to meaningful abbreviations. This will avoid file name
conflicts for module names that are truncated to the same substring.
Any module name and file name conflicts will be reported.

If your ASN.1 modules have syntactic or semantic errors, each error
will be printed to \verb$stderr$ along with the file name and line number of
where it occurred. These errors are usable by GNU emacs compiling
tools.  See the next chapter for more information on the types of
errors snacc can detect.

More errors can be detected and reported in a single compile if type
and value definitions are separated by semi-colons.  Separating type
and value definitions with semi-colons is not required, and if used,
need not be used to separate all type and value definitions.
Semi-colons are necessary after some macros that introduce ambiguity.
In general, if you get a parse error you can't figure out, try
separating the surrounding type/value definitions with semicolons.


\subsection{Known Bugs}

Snacc has problems with the following case:

\begin{verbatim}
Foo ::= SEQUENCE
{
    id IdType,
    val ANY DEFINED BY id
}

IdType ::= CHOICE
{
    a INTEGER,
    b OBJECT IDENTIFIER
}
\end{verbatim}

The error checking pass will print an error to the effect that the id
type must be INTEGER or OBJECT IDENTIFER\@.  To fix this you must modify
the error checking pass as well as the code generation pass.  To be
cheap about it, disable/fix the error checking and hand modify the
generated code.

The hashing code used for handling ANY DEFINED BY id to type mappings
will encounter problems if the hash table goes more than four levels
deep (I think this is unlikely).  To fix this just add linear chaining
at fourth level.

On the deficiency side of things, the C++ classes really need to have
free methods defined for them. (Unless you have replaced new with
something like nibble memory)

\section{\label{old-bug-section}Reporting Bugs and Your Own Improvements}

This (1.1) is the final release of snacc (I have finished my M.Sc).
Gerald Neufeld \verb$<neufeld@cs.ubc.ca>$ was my supervisor but he
does not have time to deal with support (it is all my code anyway).
Luckily, a colleague has kindly offered to receive the bug reports and
to coordinate work done by others (i.e.  you).  His name is Barry
Brachman \verb$<brachman@cs.ubc.ca>$.  He did not write the code
(35,000+ lines of C) but he has used snacc for X.500 work.  He may be
able to point you to someone who has fixed or encountered the same
bug.  Anyway, be nice to him, it's not his job.

Even though this is the second release of snacc, bugs are still
likely.  In fact, this release was quite rushed so there are probably
lots of stupid installation bugs etc.  If you find some bugs or have
other comments, please send email to \verb$snacc-bugs@cs.ubc.ca$
(these will get to Barry and Gerald).  Please include the offending
ASN.1 source, the command line options you were using and the hardware
and operating system configuration.

If you are really keen and hack in new goodies, please share.  Send
them to Barry or \verb$snacc-bugs@cs.ubc.ca$. Look in
\verb$snacc/README.future$ for things you could work on.

As I mentioned, I have entered the real world. I am now working with
Open Systems Solutions (based in New Jersey).  If your application
needs a commercially developed and supported ASN.1 compiler, try
calling 1-609-987-9073 (Yeah, I know this is a plug, but it's a good
company).
